\documentclass[1pt]{beamer}
\mode<presentation>
{
	\usetheme[usetitleprogressbar,nosmallcapitals]{m}
	\setbeamercovered{transparent}
	%\setbeamertemplate{caption}[numbered]
}
%\usepackage[T1]{fontenc}
\usepackage[francais]{babel}
\usepackage[utf8]{inputenc}
%\usepackage{lmodern}
%\usepackage{times}
\usepackage{graphicx}
\usepackage{fancyvrb}
\usepackage{amsmath}
\usepackage{amssymb}
\makeatletter
\def\verbatim{\scriptsize\@verbatim \frenchspacing\@vobeyspaces \@xverbatim}
\makeatother

	
\title{Introduction à \LaTeX{}}

\author[]{Internet -- Le \textsc{HAUM} \inst{1} -- Mathieu \textsc{Gaborit} \inst{1,}\inst{2} \\Baptiste \textsc{Ereau} \inst{1,}\inst{2}}

\institute[]
{
	\inst{1}%
	Hackerspace AU Mans ~~\url{http://www.haum.org}
		\and
	\inst{2}%
	Université du Maine
}

\date{}
\newif\iflattersubsect

\AtBeginSection[]{
	\begin{frame}<beamer>{Sommaire}
		\tableofcontents[currentsection]
	\end{frame}
	\lattersubsectfalse
}
\AtBeginSubsection[]{
    \iflattersubsect
	\begin{frame}<beamer>{Sommaire}
		\tableofcontents[currentsection, currentsubsection]
	\end{frame}
	\fi
    \lattersubsecttrue
}


\begin{document}

\begin{frame}
	\titlepage
\end{frame}


\begin{frame}{Au menu}
	\tableofcontents
\end{frame}

%%%%%%%%%%%%%%%%%%%
\section{Introduction}
\subsection{\LaTeX, Kézako ?}
%%%%%%%%%%%%%%%%%%%

%--------------------------------------------
\begin{frame}{\LaTeX ~c'est quoi ?}
	\LaTeX~ = {\em Processeur de documents} :

	\begin{itemize}
	\item texte $\rightarrow$ document (pas de forme)
	\item un poil de programmation
	\item tous types de documents (ce diapo, des partitions, des thèses ou la plupart des cours donnés par les professeurs sont \em{made in }\LaTeX ~!)
	\end{itemize}
	
	\begin{block}{}
	\centering
	Outil très puissant !
	\end{block}
	
\end{frame}

%--------------------------------------
\begin{frame}{Possibilités, Avantages}
Pourquoi \LaTeX ?
\begin{itemize}
\item Normalisation, professionnel,
\item structures complexes produites facilement (table des matières, renvois, notes de bas de page),
\item maths,
\item pléthore d'extensions pour tous les besoins,
\item communautaire et libre.
\end{itemize}
\end{frame}

%----------------------------------------
\begin{frame}{Inconvénients, Difficultés}
Un tout petit peu de bémols (parce qu'il faut bien en mettre...) :
\begin{itemize}
\item obscur au début (comme toujours),
\item difficultés à personnaliser (mais beaucoup de \textit{templates}),
\item si problème de compilation : pas de document,
\item informations souvent en ligne.
\end{itemize}
\end{frame}


%%%%%%%%%%%%%%%%%%%%%%
\section{Les bases}
\subsection{Installer / Utiliser}
%%%%%%%%%%%%%%%%%%%%%%%%%%%%%%%%%%
\begin{frame}[fragile]{Installation}
Installation et paquets :\\
\begin{center}
\begin{tabular}{|l|c|}
\hline
Linux & \texttt{\$ sudo apt-get install texlive-full}  \\\hline
Mac & Mac\TeX \\\hline
Windows & MiK\TeX ~(suivant, suivant, suivant, terminer)\\\hline
\end{tabular}
\end{center}
Création d'un document :\\
\verb|fichier.tex| $\rightarrow$ compilateur $\rightarrow$ \verb|fichier.pdf|
\end{frame}



%%%%%%%%%%%%%%%%%%%%%%
\subsection{Documents}
%%%%%%%%%%%%%%%%%%%%%%

%--------------------
\begin{frame}[fragile]{Exemple}
\begin{columns}[T]
\begin{column}{.40\textwidth}
\underline{Visuel approximatif :}\\
\begin{center}
\begin{large}
H2G2 (sur Wikipedia)
\end{large}
\begin{normalsize}
Douglas Adams\\
1982
\end{normalsize}
\end{center}
\begin{small}
Arthur Dent, citoyen anglais moyen...
\end{small}
\end{column}
\hfill
\begin{column}{.56\textwidth}
\underline{Code :} 
\begin{verbatim}
\documentclass[11pt, a4paper]{article}
% Modules à utiliser
\usepackage{url}
\usepackage[french]{babel}
\usepackage[T1]{fontenc}
\usepackage[utf8]{inputenc}

% Titre
\title{H2G2 (sur Wikipedia)}
\author{Douglas Adams}
\date{1982}

\begin{document}
    \maketitle % on affiche le titre

Arthur Dent, citoyen anglais moyen...

\end{document}
\end{verbatim}
\end{column}
\end{columns}
\end{frame}
%---------------------------------------------------------
\begin{frame}[fragile]{Explications : Préambule}
Différents types d'instructions : 
\begin{itemize}
\item Les commentaires : \%
\item Des zones définies par \verb|\begin{}| et \verb|\end{}|
\item Des commandes : \verb|\commande[options]{classe}|
\end{itemize}
\vspace{10pt}
Préambule, la première personnalisation : \begin{itemize}
\item modules \verb|\usepackage[options]{nom du paquet}|
\item options de contenu (titre, date, auteur, etc...)
\item (re)définitions de commandes \verb|\newcommand{nom}{effet}|, \verb|\renewcommand{ancienne commande}{nouvel effet}|
\end{itemize}
\end{frame}

%---------------------------------------------------
\begin{frame}[fragile]{Corps du document}
Paragraphes : 
\begin{itemize}
\item pas de marquage précis : taper !
\item \verb|\\| pour un retour à la ligne,
\item passer une ligne pour un nouveau paragraphe.
\end{itemize}

Hiérarchie des titres : part, chapter, section, subsection, subsubsection, paragraph.

Alignements : \verb|flushleft, flushright, center|, s'utilisent comme les autres avec \verb|\begin{} \end{}|.

Police : Plus {\Large gros} ou plus {\footnotesize petit}. En \textit{italique}, en \textbf{gras} ou \underline{souligné}. (resp. \verb|{\Large }|, \verb|{\footnotesize }| \verb|\textit{}| \verb|\textbf{}| \verb|\underline{}|)
\end{frame}


\subsection{Instructions plus avancées}
%%%%%%%%%%%%%% LISTES %%%%%%%%%%%%%%%
\begin{frame}[fragile]{Level up : listes}
\begin{columns}[T]
\begin{column}{.48\textwidth}
~\\
\begin{itemize}
\item Des listes 
\begin{enumerate}
\item De 
\item différents
\item types
\begin{itemize}
\item[-] parfois
\item[-] emmêlées !
\end{itemize}
\item mais
\item toujours
\end{enumerate}
\item de bon goût !
\end{itemize}
\end{column}
\hfill
\begin{column}{.48\textwidth}
\begin{verbatim}
\begin{itemize}
\item Des listes 
    \begin{enumerate}
    \item De 
    \item différents
    \item types,
        \begin{itemize}
        \item[-] parfois
        \item[-] emmêlées !
        \end{itemize}
    \item mais
    \item toujours
    \end{enumerate}
\item de bon goût !
\end{itemize}
\end{verbatim}
\end{column}
\end{columns}
\end{frame}


%%%%%%%%%%%% FIGURES %%%%%%%%%%%%%%%%%%%%%%%%%
\begin{frame}[fragile]{Level up : figures}
\begin{columns}[T]
\begin{column}{.45\textwidth}
\begin{figure}
\centering
\includegraphics[width = 0.90\textwidth]{nyan.png}
\caption{Une prodigieuse figure}
\label{fig:nyan}
\end{figure}
\end{column}
%\hfill
\begin{column}{.6\textwidth}
\begin{verbatim}
\begin{figure}
\centering
\includegraphics[width = 0.98\textwidth]{
nyan.png}
\caption{Une prodigieuse figure}
\label{fig:nyan}
\end{figure}
\end{verbatim}
\end{column}
\end{columns}
\end{frame}


%%%%%%%%% TABLEAUX %%%%%%%%%%%%%%
\begin{frame}[fragile]{Level up : tableaux}
\begin{columns}[T]
\begin{column}{.50\textwidth}
\begin{figure}
    \begin{tabular}{l|c|r}
     Numéro & Nom & Type   \\\hline
     001 & Bulbizarre & Plante\\
     025 & Pikachu & Electricité\\
     150 & Mewto & Psy\\
    \end{tabular}
    \caption{\label{tab:pokemon} Quelques pokémons}
\end{figure}
\end{column}
\hfill
\begin{column}{.50\textwidth}
\begin{verbatim}
\begin{figure}
   \begin{tabular}{l|c|r}
    Numéro & Nom & Type   \\\hline
    001 & Bulbizarre & Plante\\
    025 & Pikachu & Electricité\\
    150 & Mewto & Psy\\
   \end{tabular}
   \caption{\label{tab:pokemon}
            Quelques pokémons}
\end{figure}
\end{verbatim}
\end{column}
\end{columns}
~\\N'oubliez pas qu'il y avait la figure \ref{fig:nyan} sur la diapo précédente ! (Références avec \verb|\ref{label})|
\end{frame}

%%%%%%%%%%%%%%%%%%%%%%%%%%
%\subsection{Présentations}
%%%%%%%%%%%%%%%%%%%%%%%%%
\begin{frame}[fragile]{Présentations ? Tout est pareil !}
\begin{itemize}
\item Changer de type de doc : \verb|\documentclass[]{beamer}|.
\item Différents rendus avec la personnalisation du mode de présentation (ici, le thème m)
\item Les "frame" se construisent une à une avec \verb|\begin{frame}[options]{Titre} \end{frame}|.
\end{itemize}
\end{frame}


%%%%%%%%%%%%%%%%%%%%%
\section{Les Maths !}
\subsection{Écrire}
%%%%%%%%%%%%%%%%%%%%%
\begin{frame}{Pour frimer en société}
Tout ce qu'on peut faire avec \LaTeX ~(eq. \ref{eq:poutre}) :
\begin{equation}
\phi(\vec{x_0}) = \frac{1}{\lambda_i} \iint_{disque}\phi(\vec{x}) \frac{e^{ik|\vec{x} - \vec{x_0}|}}{|\vec{x} - \vec{x_0}|} \cos{(\vec{n}.\vec{x} - \vec{x_0})}dS\label{eq:poutre}
\end{equation}
Équation de Rayleigh, champ diffracté d'une onde acoustique par une surface (disque)
\end{frame}

%-----------------------------------------------
\begin{frame}[fragile]{Comment faire ?}
\begin{itemize}
\item Dans le texte ({\em{inline}}), entouré du symbole \$ : \verb|$\exp(x)=\sum_{k=0}^{\infty}\frac{x^k}{k!}$| donne $\exp(x)=\sum_{k=0}^{\infty}\frac{x^k}{k!}$
\item Dans un environnement ({\em{display}}) \verb|\[ \]|, \verb|\begin{equation}| ou bien \verb|\begin{eqnarray}| :
\begin{equation}
\exp(x)=\sum_{k=0}^{\infty}\frac{x^k}{k!}
\end{equation}
\end{itemize}

Notez les différences dues à l'adaptation des équations à la taille de la ligne et de la numérotation des équations délimitées !
\end{frame}

\begin{frame}[fragile]{Quelques formules}
On ne peut jamais être exhaustif donc voir (\href{http://fr.wikipedia.org/wiki/Aide:Formules_TeX}{ici}), mais sachez que :
\begin{center}
\begin{tabular}{l|c|c}
 Lettres grecques & \verb|\Psi \zeta \epsilon| & $\Psi$ $\zeta$ $\epsilon$\\\hline
 Symboles divers & \verb|\iff \in \leq \forall| & $\iff \in \leq \forall$\\\hline
 Fonctions & \verb|\cos \sqrt{x} \ln| & $\cos \sqrt{x} \ln$\\\hline
 Grands opérateurs & \verb|\sum \int_a^b \bigcup| & $\sum \int_a^b \bigcup$\\\hline
 Exposant, indice & \verb|a^{b^c} a_b| & $a^{b^c}$ $a_b$\\\hline
 Lettres d'ensembles & \verb|\mathbb{R} \mathbb{N}| & $\mathbb{R}$ $\mathbb{N}$
\end{tabular}
\end{center}
\end{frame}

%%%%%%%%%%%%%%%%%%%%
\section{Conclusion}
%%%%%%%%%%%%%%%%%%%%
\begin{frame}{Conclusion}
\LaTeX{}, ça poutre :
\begin{itemize}
\item Adaptable.
\item Puissant.
\item Esthétique.
\item Pratique.
\end{itemize}


Si plusieurs intéressés : organiser une formation ?
\end{frame}
\end{document}
